% resume.tex
% Last Updated by yew1eb in 2015/9/22

\documentclass[a4paper,11pt]{article}
\pagestyle{empty}
\raggedbottom
\raggedright
\usepackage{xcolor}
\usepackage{framed}
\usepackage{tocloft}
\usepackage{etoolbox}
\robustify\cftdotfill

\usepackage{setspace}
\usepackage{fontawesome}
\usepackage{pifont}

\usepackage[colorlinks=true,linkcolor=blue,urlcolor=blue,citecolor=blue]{hyperref} 

\usepackage{enumitem}
\setitemize[1]{itemsep=1pt,partopsep=0pt,parsep=\parskip,topsep=5pt}
\setdescription{itemsep=1pt,partopsep=0pt,parsep=\parskip,topsep=5pt}

\renewcommand{\arraystretch}{1.15}


\newlength{\outerbordwidth}
\setlength{\outerbordwidth}{3pt}  % Width of border outside of title bars
\definecolor{shadecolor}{gray}{0.75}  % Outer background color of title bars (0 = black, 1 = white)
\definecolor{shadecolorB}{gray}{0.93}  % Inner background color of title bars


%--------------------------------------------
% Chinese support
\usepackage{xeCJK}
\setCJKmainfont[BoldFont = SimHei]{楷体}
\setCJKsansfont{黑体}
\setCJKmonofont{楷体}
%--------------------------------------------


%-----------------------------------------------------------
%Margin setup
\setlength{\evensidemargin}{-0.25in}
\setlength{\oddsidemargin}{-0.25in}

\setlength{\paperwidth}{8.6in}
\setlength{\textwidth}{7.1in}

\setlength{\textheight}{11in}
\setlength{\topmargin}{-1in}
%-----------------------------------------------------------


%---------------------------------------------------------------------------------------------------------------
%Custom commands
\newcommand{\resheading}[1]{
  \parbox{\textwidth}{\setlength{\FrameSep}{\outerbordwidth}
    \begin{shaded}
\setlength{\fboxsep}{0pt}\framebox[\textwidth][l]{\setlength{\fboxsep}{3pt}\fcolorbox{shadecolorB}{shadecolorB}{\textbf{\sffamily{\mbox{~}\makebox[6.882in][l]{\large #1} \vphantom{p\^{E}}}}}}
    \end{shaded}
  }\vspace{-10pt}
}

\newcommand{\ressubheading}[4]{
	\begin{tabular*}{6.5in}{l@{\cftdotfill{\cftsecdotsep}\extracolsep{\fill}}r}
		\textbf{#1} & \textit{#2} \\
		\textit{#3} & \textit{#4} \\
	\end{tabular*}
	\vspace{-10pt}
}
%---------------------------------------------------------------------------------------------------------------



\begin{document}
\begin{tabular*}{7in}{l@{\extracolsep{\fill}}r}
\textbf{\Huge 周海} {\Large/~21岁}  & {\Large\faMobilePhone}~+86~152~7432~6049  \\
{湖南省吉首市人民南路120号} - {吉首大学砂子坳校区} - {416000} & \faEnvelope~\url{yew1eb@gmail.com}  \\
\faGithub~\url{https://github.com/yew1eb}  &  \hspace{-10pt} \faLink~\url{http://blog.csdn.net/yew1eb} \\
\end{tabular*}

%%%%%%%%%%%%%%%%%%%%%%%%%%%%%%
\resheading{教育经历}
%%%%%%%%%%%%%%%%%%%%%%%%%%%%%%
\begin{itemize}
\item \hspace{-15pt} \ressubheading{吉首大学}{湖南省湘西土家族苗族自治州}{计算机科学与技术~本科学士学位}{2012 - 至今}
\end{itemize}
\vspace{5pt}
%\hspace{30pt}GPA: \textbf{3.38/4.0}


%%%%%%%%%%%%%%%%%%%%%%%%%%%%%%
\resheading{荣誉奖励}
%%%%%%%%%%%%%%%%%%%%%%%%%%%%%%
\begin{center}
\begin{tabular*}{6.6in}{l@{\extracolsep{\fill}}r}
		\multicolumn{2}{c}{第39 届ACM 国际大学生程序设计竞赛亚洲区域赛上海站 \cftdotfill{\cftdotsep} \textbf{银奖}}\\
		\multicolumn{2}{c}{第39 届ACM 国际大学生程序设计竞赛亚洲区域赛西安站 \cftdotfill{\cftdotsep} \textbf{银奖}}\\
		\multicolumn{2}{c}{湖南省第十届大学生计算机程序设计大赛 \cftdotfill{\cftdotsep} \textbf{一等奖(冠军)}}\\
		\multicolumn{2}{c}{2014年ACM/ICPC 湖南省程序设计邀请赛 \cftdotfill{\cftdotsep} \textbf{银奖}}\\
		\multicolumn{2}{c}{湖南省第九届大学生计算机程序设计竞赛 \cftdotfill{\cftdotsep} \textbf{二等奖}}\\
		\multicolumn{2}{c}{2013–2014学年~吉首大学优秀学生奖学金 \cftdotfill{\cftdotsep} \textbf{三等奖}}\\
		\multicolumn{2}{c}{2012–2013学年~吉首大学优秀学生奖学金 \cftdotfill{\cftdotsep} \textbf{一等奖}}\\
		\vphantom{E}
\end{tabular*}\vspace{-25pt}
\end{center}



%%%%%%%%%%%%%%%%%%%%%%%%%%%%%%
\resheading{个人经历}
%%%%%%%%%%%%%%%%%%%%%%%%%%%%%%
\begin{itemize}
	\item 高中:参加了两届NOIP竞赛,学习并掌握了基本的数据结构和算法
	\item 大二:组建ACM集训队,并担任第一任集训队长。大三:带领队伍首次打进亚洲区域赛
\end{itemize}



%%%%%%%%%%%%%%%%%%%%%%%%%%%%%%
\resheading{项目实践}
%%%%%%%%%%%%%%%%%%%%%%%%%%%%%%
\begin{itemize}
\item \hspace{-15pt} \ressubheading{\textbf{Online Judge判题端开发},\url{https://github.com/yew1eb/OnlineJudgeCore}}{}{}{} \\
这个项目是Online Judge(在线程序测评系统)的核心部分,此系统被用来帮助校ACM/ICPC队伍组织和管理各种训练计划和相关比赛,还将运用于大学生计算机课程上机实验。\\
职责:独立开发

\vspace{10pt}
\item \hspace{-15pt} \ressubheading{\textbf{CrawlNews},\url{https://github.com/yew1eb/CrawlNews}}{}{}{} \\
可部署到Linux 服务器上,实时爬取教务处公告,判断是否有新通知,并把新通知推送到指定邮件的简单Python 爬虫程序
\end{itemize}

%%%%%%%%%%%%%%%%%%%%%%%%%%%%%%
\resheading{专业技能}
%%%%%%%%%%%%%%%%%%%%%%%%%%%%%%
\begin{itemize}
\item[-] 熟悉C/C++ 编程,有Java, Python, Php 开发经验
\item[-] 熟练使用Linux操作系统,了解linux下的网络编程和多线程编程
\item[-] 掌握Hadoop集群的搭建和配置,理解HDFS和MapReduce的设计原理
\item[-] 目前正在跟进MOOC课程《机器学习》和《机器学习基石》,并在博客中做笔记记录
\end{itemize}



%%%%%%%%%%%%%%%%%%%%%%%%%%%%%%
\resheading{自我评价}
%%%%%%%%%%%%%%%%%%%%%%%%%%%%%%
\begin{itemize}
\item[-] 能在较短时间内独立思考、周密分析问题并形成解决问题的思路;由于经常参加程序竞赛,提高了数学英语功底、提升了算法设计能力和编程技巧、养成了良好的协作精神、锻炼了心理素质和临场应变能力
\item[-] 对新技术有浓厚的兴趣,有很强的自学能力
\end{itemize}

\end{document}

























