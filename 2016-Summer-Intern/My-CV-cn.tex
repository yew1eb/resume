\documentclass[12pt,a4paper,sans]{moderncv}

%moderncv主题
\moderncvtheme[black]{banking}
% CV color - options include: 'blue' (default), 'orange', 'green', 'red', 'purple', 'grey' and 'black'
% CV theme - options include: 'casual' (default), 'classic', 'oldstyle' and 'banking'
\definecolor{color0}{rgb}{0,0,0}% black
% \definecolor{color1}{rgb}{0.95,0.20,0.20}% red
\definecolor{color2}{rgb}{0.0,0.0,0.0}% dark grey

%不显示页码
\nopagenumbers{}

%中文支持
%\usepackage{ctex}
\usepackage{xeCJK}
\setCJKmainfont{楷体}
\setCJKsansfont{黑体}
\setCJKmonofont{楷体}


%调整页面边距
\usepackage{geometry}  % 页边距
\geometry{left=1.8cm,right=1.8cm,top=1.2cm,bottom=0cm}  % 设置页边距
\thispagestyle{empty}  % 去掉默认页码
\AtBeginDocument{
\hypersetup{colorlinks,urlcolor=blue} % 超链接颜色
}

\renewcommand*{\namefont}{\fontsize{30}{30}\mdseries\upshape}
\name{周}{海}
%\title{个人简历}
\address{湖南省吉首市人民南路120号}{吉首大学砂子坳校区}{416000}
\mobile{+86~152~7432~6049}
\email{yew1eb@gmail.com}
\homepage{yew1eb.net}
%\extrainfo{additional information}   
%\photo[64pt][0.4pt]{picture}        
%\quote{Some quote (optional)}    

\begin{document}
\maketitle
\vspace{-3.4em} % 缩小段落的间距
\section{基本信息}
%\cvitem{姓名}{周海} 
%\cvitem{年龄}{21(1994-03-12)}
%\cvitem{毕业时间}{2016-06}
%\cvitem{求职意向}{研发工程师,算法工程师}

%@{\extracolsep{\fill}} 将所有列在设定的表格宽度中均匀展开
\begin{tabular*}{12cm}{@{\extracolsep{\fill}}ll} 
 姓名:周海 &籍   贯:湖南衡阳  \\
年龄:21(1994-03-12) & 毕业时间:2016-06 \\
求职意向:研发工程师,算法工程师\\
\end{tabular*}

\vspace{-0.5em}
\section{教育背景}
\cventry{2012 -- 至今}{计算机科学与技术专业}{\textsf{吉首大学}}{湖南省吉首市}{工学学士}{GPA: 3.24 / 4.0}

\vspace{-0.5em}
\section{荣誉奖励}
%\subsection{ACM-ICPC}
\cvline{2014}{第39届ACM国际大学生程序设计竞赛亚洲区域赛上海站 \hfill    \textsf{银奖}}
\cvline{2014}{第39届ACM国际大学生程序设计竞赛亚洲区域赛西安站 \hfill    \textsf{银奖}}
\cvline{2014}{湖南省第十届大学生计算机程序设计竞赛 \hfill    \textsf{一等奖(冠军)}}
\cvline{2014}{ACM/ICPC湖南省程序设计邀请赛   \hfill    \textsf{银奖}}
\cvline{2013}{第38届ACM国际大学生程序设计竞赛亚洲区域赛长沙站 \hfill   \textsf{优胜奖}}
\cvline{2013}{湖南省第九届大学生计算机程序设计竞赛 \hfill    \textsf{二等奖}}
%\subsection{其他荣誉}
\cvline{2013-2014}{吉首大学优秀学生奖学金 \hfill    \textsf{三等奖}}
\cvline{2012-2013}{吉首大学优秀学生奖学金 \hfill    \textsf{一等奖}}

\vspace{-0.5em}
\section{校园经历}
\begin{itemize}
\item[-] 高一开始学习编程,参加NOIP竞赛,虽未取得优秀成绩,但奠定了良好的编程基础
\item[-] 大一暑假组建学校ACM集训队,并担任集训队长至今,带领队伍首次打进亚洲区域赛
\end{itemize}

\vspace{-0.5em}
\section{实习经历}
\cvitem{}{张家界广电传媒有限责任公司网络媒体平台 - 爱视网 \hfill 2015/01 - 2015/02}{修复商城模块的后台支付功能,以及现实用户到店自提功能}


\vspace{-0.5em}
\section{个人作品}
\cventry{}{使用Swing实现界面,通过多线程控制鱼的游动、子弹移动}{\textsf{Java实现捕鱼达人}}{大三下学期}{}
{}
\cventry{}{使用JSP和MySQL,实现了班级信息、学生信息管理功能}{\textsf{简易学生信息管理系统}}{大二下学期}{}{}
\cventry{}{基于Windows Console, 实现了积分、难度设置和播放背景音乐功能}{\textsf{C++实现俄罗斯方块}}{大一下学期}{}{}

\vspace{-0.5em}
\section{专业技能}
\subsection{计算机技能}
\cvitem{\textsf{编程语言}}{熟悉C/C++,了解Python,Java,Php}
\cvitem{\textsf{基础技能}}{熟悉Linux的常用命令,了解多线程和Socket网络编程}
\cvitem{\textsf{开发工具}}{Vim,Eclipse,Visual Studio,MySQL,Git,Makefile}

\subsection{其它技能}
\cvitem{\textsf{问题解决}}{能在较短时间内独立思考、周密分析问题并形成解决问题的思路;由于经常参加程序竞赛,提高了数学英语功底、提升了算法设计能力和编程技巧、养成了良好的协作精神、锻炼了心理素质和临场应变能力}
%\cvitem{\textsf{学习能力}}{对新技术有浓厚的兴趣,有很强的自学能力}
%\cvitem{\textsf{团队合作}}{有良好的沟通能力和很强的团队意识}

\end{document}
































